% !Mode:: "TeX:UTF-8"
%!TEX program  = xelatex

%\documentclass{cumcmthesis}
\documentclass[withoutpreface,bwprint]{cumcmthesis} %去掉封面与编号页



\usepackage{url}
\title{\LARGE 小锅的机器学习笔记--支持向量机}


\schoolname{苏州大学}


\begin{document}
	\makeatletter %使\section中的内容左对齐
	\renewcommand{\section}{\@startsection{section}{300}{0mm}
		{-\baselineskip}{0.5\baselineskip}{\bf\leftline}}
	\renewcommand{\subsection}{\@startsection{section}{300}{5mm}
		{-\baselineskip}{0.5\baselineskip}{\bf\leftline}}
 \maketitle

%目录
%\tableofcontents
%\newpage
	\subsection{\Large 数据:}
	 \begin{align*}{ll} \\
		i_t = \sigma(W_{ii} x_t + b_{ii} + W_{hi} h_{t-1} + b_{hi}) \\
		f_t = \sigma(W_{if} x_t + b_{if} + W_{hf} h_{t-1} + b_{hf}) \\
		g_t = \tanh(W_{ig} x_t + b_{ig} + W_{hg} h_{t-1} + b_{hg}) \\
		o_t = \sigma(W_{io} x_t + b_{io} + W_{ho} h_{t-1} + b_{ho}) \\
		c_t = f_t \odot c_{t-1} + i_t \odot g_t \\
		h_t = o_t \odot \tanh(c_t) \\
	\end{align*}
	设样本为$(x_i,y_i)$,$x_i \in R_{p}$,观察数据共有$n$个样本,分别为$(x_1,y_i),\ldots,(x_n,y_n)$,且
	\begin{equation*}
		y_i=\begin{cases}
			y_i=1 \quad \quad if \quad \quad \textbf{第$i$个样本为正类}\\
			y_i=0 \quad \quad if \quad \quad \textbf{第$i$个样本为负类}\\
		\end{cases}
	\end{equation*}
	SVM的判别模型为:
	\begin{equation}
		f(x_i)=sign(W^Tx_i+b) \quad \quad sign(y) =\begin{cases}
			+1 \quad if \quad {y \geq 0}\\
			-1 \quad if \quad {y<0}
		\end{cases}
	\end{equation}
	式中$W \in R_p$,$b \in R$为SVM模型要求解的参数。
	\subsection{\Large 硬间隔SVM}
	硬间隔要求对每一个样本点分类正确,那么有约束条件:
	\begin{equation}
		 s.t. \quad y_i\left( W^Tx_i+b \right) > 0  \quad \quad i=1,\ldots,n      \label{con:st}
	\end{equation}
	在高维空间中,设第$i$个样本点到超平面$W^Tx+b$的距离为:
	\begin{equation}
		Dis_i=\dfrac{|W^Tx_i+b|}{||W||_2}
	\end{equation}
		又因为约束条件(\ref{con:st}),所以:
	\begin{equation}
		Dis_i=\dfrac{y_i\left( W^Tx_i+b \right)}{||W||_2}
	\end{equation}
	在机器学习中,追求的是泛化误差,而不是训练误差,要让泛化误差最大,也就是要让离超平面最近的那个样本点离超平面的距离更最大化,也就是最大离超平面
	最近的那个点离超平面之间的函数几何间隔。
	\begin{align*}
		W,b &= \mathop{argmax} \limits_{W,b} \mathop{min} \limits_{i=1,..,n}  Dis_i \\
			&=\mathop{argmax} \limits_{W,b}  \dfrac{1}{||W||_2} \mathop{min} \limits_{i=1,..,n} y_i\left( W^Tx_i+b \right)\\
			&=\mathop{argmax} \limits_{W,b}  \dfrac{1}{||W||_2} D
	\end{align*}
	其中$D=\mathop{min} \limits_{i=1,..,n} y_i\left( W^Tx_i+b \right)$\\
	综上,求解硬间隔的SVM就是求解一个优化问题,优化的约束与目标为:
	\begin{equation}
	\begin{cases}
		\mathop{max} \limits_{W,b} \quad \dfrac{1}{||W||_2} D \\
		s.t. \quad y_i\left( W^Tx_i+b \right) \geq D  \quad \quad i=1,\ldots,n
	\end{cases}
	\end{equation}
	 $y_i\left( W^Tx_i+b \right)$ 是点与函数之间的函数间隔,同时的让$W,b$增大$\lambda$倍变成$\lambda W,\lambda b$,此时$D,||W||_2$也增加了$\lambda$倍,对我们的优化问题没有任何的影响,所以我们让$D=1$。
	 最终优化问题可以写成:
	\begin{equation}
	\begin{cases}
		\mathop{max} \limits_{W,b} \quad \dfrac{1}{||W||_2}  \\
		s.t. \quad 1-y_i\left( W^Tx_i+b \right) \leq 0  \quad \quad i=1,\ldots,n
	\end{cases}
	\end{equation}
	因为最大化$\dfrac{1}{||W||_2}$和最小化$\dfrac{W^TW}{2}$是等价的,所以可以继续写成
	\begin{equation}
	\begin{cases}
		\mathop{min} \limits_{W,b} \quad \dfrac{W^TW}{2}  \\
		s.t. \quad 1-y_i\left( W^Tx_i+b \right) \leq 0  \quad \quad i=1,\ldots,n   \label{con:ff}
	\end{cases}
	\end{equation}
	我们可以用拉格朗日乘子法改写上式成:
	\begin{equation}
		L=\dfrac{W^TW}{2}+\sum_{i=1}^{n} \lambda_i \left[  1-y_i\left( W^Tx_i+b \right) \right] \quad \lambda_i \geq 0
	\end{equation}
	可以写成:
	\begin{equation}
		\begin{cases}
		 &\mathop{min} \limits_{W,b} \mathop{max} \limits_{\lambda_i} L
		\\ &s.t. \quad \lambda_i \geq 0 \quad i=1,\ldots n
		\end{cases} 
		\label{con:H}
	\end{equation}
	证明不会,但是这样想,如果存在一个$x_i,y_i$,使$\left[  1-y_i\left( W^Tx_i+b \right) \right] >0$,那么在优化$\mathop{max} \limits_{\lambda_i} L$时就会让$\lambda_i=+\infty$,此时$\mathop{min} \limits_{W,b} \mathop{max} \limits_{\lambda_i} L=+\infty$,这样就没有任何意义了。\\
	所以$\left[  1-y_i\left( W^Tx_i+b \right) \right] \leq 0 \quad i=1,\ldots n$衡成立。\\
	并且在优化$\mathop{max} \limits_{\lambda_i} L$时,会让$\lambda_i$全部为0,那么我们最后就在求$\mathop{min} \limits_{W,b} \dfrac{W^TW}{2} $。\\
	由于这是一个二次凸优化问题,存在强对偶关系,即:
	\begin{equation}
		\mathop{min} \limits_{W,b} \mathop{max} \limits_{\lambda_i} L=\mathop{max} \limits_{\lambda_i} \mathop{min} \limits_{W,b} L
	\end{equation}
	首先我们求解$\mathop{min} \limits_{W,b} L$:
	\begin{align*}
			& \frac{\partial L}{\partial W}=W-\sum_{i=1}^{n}\lambda_i y_i x_i \\
			& \frac{\partial L}{\partial b}=-\sum_{i=1}^{n} \lambda_i y_i
	\end{align*}
	令$\frac{\partial L}{\partial W},\frac{\partial L}{\partial b}$等于$0$,有:。
	\begin{equation}
		\begin{cases}
			W=\sum_{i=1}^{n}\lambda_i y_i x_i \\
			\quad	\sum_{i=1}^{n}\lambda_i y_i=0
		\end{cases}
	\label{con:wb}
	\end{equation}
	将(\ref{con:wb})带入$L$中可以得到:
	\begin{equation}
		\mathop{min} \limits_{W,b} L=\dfrac{1}{2}(\sum_{i=1}^{n}\lambda_i y_i x_i)^T (\sum_{i=1}^{n}\lambda_i y_i x_i)+\sum_{i=1}^{n}\lambda_i \left[1-y_iW^Tx_i  \right]
	\end{equation}
	又因为:
	\begin{align*}
		& (\sum_{i=1}^{n}\lambda_i y_i x_i)^T (\sum_{i=1}^{n}\lambda_i y_i x_i)
		\\&=
		\left[
		\begin{array}{cccc}
			\lambda_1 y_1,\lambda_2 y_2,\ldots,\lambda_n y_n
		\end{array}
		\right] \left[
			\begin{array}{c}
				x_1^T \\
				x_2^T \\
				\vdots \\
				x_n^T
			\end{array}
		\right] \left[
		\begin{array}{cccc}
			x_1,x_2,\ldots,x_n
		\end{array}
		\right] \left[
			\begin{array}{c}
				\lambda_1 y_1\\
				\lambda_2 y_2\\
				\vdots\\
				\lambda_n y_n
			\end{array}
		\right]
		\\\\
		&=
			\left[
		\begin{array}{cccc}
			\lambda_1 y_1,\lambda_2 y_2,\ldots,\lambda_n y_n
		\end{array}
		\right] 
		\left[
			\begin{array}{cccc}
				x_1^Tx_1,&x_1^Tx_2,&\ldots,&x_1^T x_n\\
				x_2^Tx_1,&x_2^Tx_2,&\ldots,&x_2^T x_n\\
				\vdots &\vdots &\vdots &\vdots \\
				x_n^Tx_1,&x_n^Tx_2,&\ldots,&x_n^T x_n\\	
			\end{array}
		\right]
		 \left[
		\begin{array}{c}
			\lambda_1 y_1\\
			\lambda_2 y_2\\
			\vdots\\
			\lambda_n y_n
		\end{array}
		\right]				
	\end{align*}
	这就是一个二次型,所以:
	\begin{equation}
		(\sum_{i=1}^{n}\lambda_i y_i x_i)^T (\sum_{i=1}^{n}\lambda_i y_i x_i)=\sum_{i=1}^{n}\sum_{j=1}^{n}\lambda_i \lambda_j y_i y_j x_i^T x_j
	\end{equation}
	所以有:
	\begin{equation}
			\mathop{min} \limits_{W,b} L=\dfrac{1}{2} \sum_{i=1}^{n}\sum_{j=1}^{n}\lambda_i \lambda_j y_i y_j x_i^T x_j+\sum_{i=1}^{n}\lambda_i \left[1-y_iW^Tx_i  \right]
	\end{equation}
	最后我们再把$W^T=\sum_{i=1}^{n}\lambda_iy_ix_i^T$带入:
	\begin{equation}
			\mathop{min} \limits_{W,b} L=\sum_{i=1}^{n}\lambda_i -\dfrac{1}{2} \sum_{i=1}^{n}\sum_{j=1}^{n} \lambda_i \lambda_j y_i y_j x_i^T x_j
	\end{equation}
	所以我们的最终需要优化的模型是:
	\begin{equation}
		\begin{cases}
				\mathop{max} \limits_{\lambda_i} \quad \sum_{i=1}^{n}\lambda_i -\dfrac{1}{2} \sum_{i=1}^{n}\sum_{j=1}^{n} \lambda_i \lambda_j y_i y_j x_i^T x_j
				\\
				\quad \quad s.t.\quad \quad \lambda_i \geq 0 \quad \quad i=1,\ldots,n \\
				\quad \quad \quad \quad \quad \sum_{i=1}^{n} \lambda_i y_i=0
		\end{cases}
	\end{equation}
	这个模型可以用SMO算法求解出$(\lambda_1,\lambda_2,\ldots,\lambda_n)$。
	剩下就是求解$W$和$b$,因为(\ref {con:H}) 存在强对偶关系,所以必定满足KKT条件,即:
	\begin{equation}
		\begin{cases}
			\frac{\partial L}{\partial W}=0 \quad \quad \frac{\partial L}{\partial b}=0\\
			\lambda_i\left[1-y_i(W^Tx_i+b)\right]=0 \quad \quad i=1,\ldots,n \\
			\lambda_i \geq 0 \quad \quad i=1,\ldots,n \\
			1-y_i(W^Tx_i+b) \leq 0
		\end{cases}
	\end{equation}
	当$1-y_i(W^Tx_i+b) < 0$是,如果要使$\lambda_i\left[1-y_i(W^Tx_i+b)\right]=0$,那么此时一定有$\lambda_i=0$。
	我们称满足$1-y_i(W^Tx_i+b)=0$的$(x_i,y_i)$为支持向量,对于支持向量,$\lambda_i \geq 0$。\\
	支持向量一定存在,因为这个所谓的$1$是我们取的,本质上这个$1$的指的就是离分类超平面最近的那个向量与超平面的距离,所以我们至少存在一个支持向量(离超平面最近的那个样本)。\\
	设第$k$个样本为支持向量,$W=\sum_{i=1}^{n}\lambda_iy_ix_i$那么有:
	\begin{equation*}
		y_k(W^Tx_k+b)=1 \quad \Longleftrightarrow \quad b_k=y_k-(\sum_{i=1}^{n}\lambda_iy_ix_i^T)x_k
	\end{equation*}
	设支持向量的集合为$S$,且对于非支持向量$x_f,y_f$,它们对应的$\lambda_f$等于0,那么:
	\begin{equation}
		\begin{cases}
			W & =\quad \sum_{x_k \in S} \lambda_ky_kx_k  \\
			b & =\quad \dfrac{1}{|S|}\sum_{x_k \in s} \left[ y_k-(\sum_{i=1}^{n}\lambda_iy_ix_i^T)x_k \right]
		\end{cases}
	\end{equation}
	我们可以发现无论是$W$还剩$b$都只与支持向量有关,这也是SVM被称为支持向量机的原因。
	
	\subsection{\Large 软间隔SVM}
	通常,观测的数据存在噪声,也就是指有的正类样本被观察成了负类,或者原本数据就不是线性可分的,但是我就是要用线性分开,这种情况下硬间隔的SVM就不顶用了,这个时候就需要软间隔的SVM,软间隔的SVM和硬间隔的思想一样,但是允许样本被分类错误,在(\ref{con:ff})的基础上,记对样本$i$的分类错误损失为:
	\begin{equation}
		Loss_i=max\left\{0,1-y_i(W^Tx_i+b)\right\}
	\end{equation}
	如果$1-y_i(W^Tx_i+b)$小于$0$,那么这个时候样本$i$分类正确,此时$Loss_i$为$0$,但是如果分类错误即$1-y_i(W^Tx_i+b)$大于$0$,此时$Loss_i$就是损失。\\
	对于第$i$个样本的约束条件改写为:
	\begin{equation}
		y_i\left(W^Tx_i+b\right)\geq 1-Loss_i
	\end{equation}
	改成这样的原因是因为:\\
	 \indent  如果第$i$个样本分类真确,$Loss_i$为0,此时约束条件不变。\\
	 \indent  如果第$i$个样本分类错误,$Loss_i=1-y_i(W^Tx_i+b)$,此时约束不等式为衡等式,依旧满足。\\
	 所以(\ref{con:ff})改写成:
	 \begin{equation}
	 	\begin{cases}
	 		\mathop{min} \limits_{W,b} \quad \dfrac{W^TW}{2} +C\sum_{i=1}^{n} Loss_i \\
	 		s.t. \quad 1-y_i\left( W^Tx_i+b \right)-Loss_i \leq 0  \quad \quad i=1,\ldots,n  
	 	\end{cases}
 		\label{con:asd1}
	 \end{equation}
 	$C$为一个超参数,$C$越大,软间隔SVM就越偏向把训练集中数据全部分类正确,$C$越小,就偏向允许以训练集某些样本分类错误为代价去追去更大的间隔。\\
 	\indent 但是直接这样是没法搞的,我们引入松弛变量$\theta_i$,修改上式为:
 	\begin{equation}
 		\begin{cases}
 			\mathop{min} \limits_{W,b,\theta_i} \quad \dfrac{W^TW}{2} +C\sum_{i=1}^{n} \theta_i \\
 			s.t. \quad 1-y_i\left( W^Tx_i+b \right)-\theta_i \leq 0  \quad \quad i=1,\ldots,n \\
 			\quad \quad \quad \theta_i \geq 0  
 		\end{cases}
 		\label{con:asd2}
 	\end{equation}
 	\indent 当样本$i$分类正确时,有$1-y_i\left( W^Tx_i+b \right)\leq 0$,我们追求目标最小化,此时会使$\theta_i=0$,此时无损失。\\
 	\indent 当样本$i$分类错误时,有$1-y_i\left( W^Tx_i+b \right)> 0$,因为我们追求目标最小化,此时我们希望$\theta_i$尽可能的小,但是需要满足$1-y_i\left( W^Tx_i+b \right)-\theta_i \leq 0$,那么此时会使$\theta_i=1-y_i\left( W^Tx_i+b \right)$。\\
 	所以引入松弛变量$\theta_i$后的优化目标(\ref{con:asd2})和原始(\ref{con:asd1})是一样的。\\
 	我们引入拉格朗日乘子,可以得到我们的优化目标函数为:
 	\begin{equation}
 		L=\dfrac{W^TW}{2} +C\sum_{i=1}^{n} \theta_i+\sum_{i=1}^{n}\lambda_i \left[ 1-y_i\left( W^Tx_i+b \right)-\theta_i \right]
 		-\sum_{i=1}^{n} \mu_i \theta_i \label{con:L1}
 	\end{equation}
 	其中$\mu_i,\theta_i$均大于等于0。
 	同硬间隔一样,软间隔可以写成:
 		\begin{equation}
 		\begin{cases}
 			&\mathop{min} \limits_{W,b,\theta_i} \mathop{max} \limits_{\lambda_i} L
 			\\ &s.t. \quad \lambda_i,\mu_i \geq 0 \quad i=1,\ldots n
 		\end{cases} 
 	\end{equation}
 	由于是二次凸优化问题,所以存在强对偶关系,上式可以写成:
 	\begin{equation}
 		\begin{cases}
 			& \mathop{max} \limits_{\lambda_i} \mathop{min} \limits_{W,b,\theta_i} L
 			\\ &s.t. \quad \lambda_i,\mu_i \geq 0 \quad i=1,\ldots n
 		\end{cases} 
 	\end{equation}
 	
 	
 	
 	对$L$分别求$W,b,\theta_i$的偏导数并且令其均等于$0$:
 	\begin{equation}
 		\begin{cases}
 			& \frac{\partial L}{\partial W}=W-\sum_{i=1}^{n}\lambda_i y_i x_i \\
 			& \frac{\partial L}{\partial b}=-\sum_{i=1}^{n} \lambda_i y_i   \\
 			& \frac{\partial L}{\partial \theta_i}=C-\lambda_i-\mu_i
 		\end{cases}
 	\end{equation}
 	首先我们带入$C=\lambda_i+\mu_i$进入(\ref{con:L1}),可以得到:
 	\begin{align*}
 		L & =\dfrac{W^TW}{2}+\sum_{i=1}^{n}\lambda_i \left[ 1-y_i\left( W^Tx_i+b \right)\right]\\
 	\end{align*}
 	 \indent 带入 $W=\sum_{i=1}^{n}\lambda_i y_i x_i \quad \sum_{i=1}^{n} \lambda_i y_i=0$,有:
 	 \begin{equation*}
 	 	L= \sum_{i=1}^{n}\lambda_i -\dfrac{1}{2} \sum_{i=1}^{n}\sum_{j=1}^{n} \lambda_i \lambda_j y_i y_j x_i^T x_j
 	 \end{equation*}
  		因为$\mu_i \geq 0,C=\lambda_i+\mu_i$,并且我们带入后的$L$没有$\mu_i$,所以我们相较于硬间隔的SVM多了一个约束条件:
  	\begin{equation}
  			0 \leq \lambda_i \leq C \quad \quad i=1,\ldots n
  	\end{equation}
  	所以我们软间隔的优化函数可以写成:
  		\begin{equation}
  		\begin{cases}
  			\mathop{max} \limits_{\lambda_i} \quad \sum_{i=1}^{n}\lambda_i -\dfrac{1}{2} \sum_{i=1}^{n}\sum_{j=1}^{n} \lambda_i \lambda_j y_i y_j x_i^T x_j
  			\\
  			\quad \quad s.t.\quad \quad  0 \leq \lambda_i \leq C \quad \quad i=1,\ldots,n \\
  			\quad \quad \quad \quad \quad \sum_{i=1}^{n} \lambda_i y_i=0\\
  		\end{cases}
  	\end{equation}
  	这个依旧可以利用SMO算法求解。\\
  	最后我们求$W,b$:
  	因为该问题存在强对偶关系,所以满足KKT条件:
  	\begin{equation}
  		\begin{cases}
  			W=\sum_{i=1}^{n}\lambda_i y_i x_i \quad \sum_{i=1}^{n} \lambda_i y_i=0 \quad C=\lambda_i+\mu_i \\
  			\mu_i \geq 0 \quad \quad \lambda_i \geq 0 \\
  			1-y_i(W^Tx_i+b)-\theta_i \leq 0 \quad \quad  \theta_i \geq 0 \\
  			\lambda_i\left[  1-y_i(W^Tx_i+b) -\theta_i ) \right]=0 \quad \quad \mu_i\theta_i=0\\ 			
  		\end{cases}
  	\end{equation}
  	显然有$W=\sum_{i=1}^{n}\lambda_i y_i x_i$,其次若存在一个$ 0<\lambda_k<C$,此时一定有$u_k=C>0$,则$\theta_k=0$,那么就是$1-y_k(W^Tx_k+b)=0$,所以满足$0<\lambda_k<C$的样本就是支持向量,所以:
  	\begin{equation}
  		b=y_k-W^Tx_i
  	\end{equation}
  \subsection{\Large 核方法:}
  	有些数据集就不是线性可分的,那么就算是使用了软间隔的SVM这个数据集也没法做,那么解决的方法是把低维度线性不可分的数据投影到高维度空间,那么就有可能可以分开。\\
  	设$R_1$是输入空间,$R_2$是特征空间,如果存在一个从$R_1$到$R_2$的映射:
  	\begin{equation}
  		\phi(x):R_1 \to R_2
  	\end{equation},使得对于所有的$x,z \in R_1$。函数$K(x,z)$都满足条件:
  \begin{equation}
  		K(x,z)=\phi(x) \cdot \phi(z)
  \end{equation}
  	则称呼$K(x,z)$为核函数。
  	引入核函数后SVM的优化函数变成了:
  	\begin{equation}
  			\begin{cases}
  			\mathop{max} \limits_{\lambda_i} \quad \sum_{i=1}^{n}\lambda_i -\dfrac{1}{2} \sum_{i=1}^{n}\sum_{j=1}^{n} \lambda_i \lambda_j y_i y_j \phi(x_i^T) \phi( x_j)
  			\\
  			\quad \quad s.t.\quad \quad  0 \leq \lambda_i \leq C \quad \quad i=1,\ldots,n \\
  			\quad \quad \quad \quad \quad \sum_{i=1}^{n} \lambda_i y_i=0\\
  		\end{cases}
  	\end{equation}
  	并且$W,b$变成了:
  	\begin{equation}
  		\begin{cases}
  			W=\sum_{i=1}^{n}\lambda_i y_i \phi(x_i)\\
  			b=y_k-W^T\phi(x_i)=y_k-\sum_{j=1}^{n}\lambda_j y_j \phi(x_j^T) \phi(x_i)=y_k-\sum_{j=1}^{n}\lambda_j y_j K(x_i,x_j) 
  		\end{cases}
  	\end{equation}
  	引入和函数后,$W$就不能显式的求出来了,但是我们观察到,当我们要做预测$x_k$的时候需要计算$W^T\phi(x_k)$,可以拆解为:
  	\begin{equation}
  		W^T\phi(x_k)=\sum_{i=1}^{n}\lambda_i y_i \phi(x_i) \phi(x_k) =\sum_{i=1}^{n}\lambda_iy_i K(x_i,x_k)
  	\end{equation}
  	
  	
  	
  	
 
 
 	
 	
 	
 	
	
	
	
	
	
	
	
\end{document}
	

